% !TEX root = ../../main.tex

We have previously seen the challenges of transformation learning.
We will now study the state of the art for learning transformations from
a set of examples.
We will mainly focus on fully automatic approaches learning API usage updates
from examples, and assess the strengths and weaknesses of these approaches
in terms of the taxonomy challenges we introduced earlier.
We will also see more distant related work, with semi-automatic approaches
and approaches learning transformations other than API usage updates.


\section{TBD}

Much previous work on data in edge computing 
\cite{app:rep:1826, syn:optim:rep:1433, rep:1844}
focuses on streaming and content delivery.

Examples include sensor systems or propagating database views.
We leverage this previous work by propagating shared state as a
stream of update events.

The distributed sharing of persistent mutable state raises extra
challenges, which we address in this paper.

Achieving low response time is an ongoing challenge for many web applications
\cite{akamai:2010:press, kohavi2007online, leighton2009improving}.
For example, on amazon.com, a delay of 100 ms costs in average 1\% of
sales \cite{kohavi2007online}.
In order to deliver fast response and offline support, a number of web
applications cache data at the client side, e.g., in the browser.
See for instance Facebook's News Feed \cite{facebook:newsfeed}, or the
Chrome browser extension that enables Google Docs and Google Maps to be
used offline \cite{googledocs:offline}.
Several earlier systems manage data for clients that are connected only
intermittently \cite{optim:rep:syn:1500,rep:syn:app:1728}.
Bayou \cite{syn:optim:rep:1433} supports data replicas in
mobile computers.
Cimbiosys \cite{app:rep:1826} uses a decentralised model to support
Internet Services.
Rover \cite{rep:1844} and Coda \cite{fic:rep:887} are general-purpose
systems that support disconnected operation, under a weak consistency
model.


Cloud Types \cite{syn:app:1761} support a programming
model for shared cloud data, similar to CRDTs with total order.
It supports multiple replicas of data, synchronised periodically.
Parse \cite{syn:app:1761} is similar, under a weaker eventual consistency
model.

% CC, CC+ and TCC+
Strengthening causal consistency with strong convergence was proposed by
COPS under the name Causal Plus Consistency (CC+) \cite{rep:syn:1662}.
ChainReaction augments CC+ with transactional reads, implemented with
the help of a sequencer per DC.
To execute a write, ChainReaction requires that the prior versions read at the client are stable in the DC.
%
TCC+ \cite{rep:pro:sh182} extends the guarantees of CC+ to transactions.
This model is closely related to others, such as Parallel Snapshot
Isolation (PSI) proposed by the Walter system \cite{rep:syn:1661}.
TCC+ is the strongest model compatible with availability under partition.
%It does order concurrent transactions, but only requires convergence.
\system{} guarantees TCC+ globally; in zones with good connectivity,
it enforces an SI zone to
improve metadata overhead and user experience.
Whereas TCC+ supports concurrent updates and arbitrary CRDT types, PSI
restricts concurrency to a single data type (the cset).
Its SI zones, the DCs, are fixed in advance.
Walter runs two-phase commit across data centres; this ensures global
transactions are strongly consistent, but negatively impacts liveness
and performance.

% hybrid consistency models
A hybrid consistency model makes different consistency guarantees for
different operations.
For instance, in Lazy Replication, operations are causally consistent by
default, but can optionally request linearisability \cite{pan:rep:1146}.
References
\citealp{alg:rep:syn:optim:1464,rep:syn:sh167} or \citealp{rep:syn:1690}
are other examples.
%
Unistore \cite{unistore} supports both causal and linearisable
transactions over a geo-replicated key-value store.
To ensure fault tolerance, Unistore requires that, before a linearisable
transaction commits, all of its causal dependencies be stable.
%
\citet{fisheye} propose a proximity-based hybrid model, called Fisheye
Consistency, such that nodes that are close in a ``proximity graph''
mutually observe strongly-consistent transactions, whereas consistency
is weak between far-away nodes.

Depot \cite{rep:1712} and PRACTI \cite{rep:syn:1729} pioneered
highly available caching at the edge, under causal consistency.
Their approach uses vector clocks with one entry per replica, which severely
limits scalability.
Depot targets Byzantine fault tolerance, the most general class of
faults, but not transactions.
Simba \cite{fic:syn:1731} enables the edge application to select among a
specific level of consistency (eventual, causal or serialisable).
PouchDB \cite{pouchDB} is a client-side cache that replicates data
from a CouchDB server; it supports offline operation and detects
conflicts, but does not merge them.

SwiftCloud \cite{rep:pan:sh177} introduces bounded-size
vectors and migration.
Legion \cite{app:rep:syn:1775} is a web framework that extends applications
with peer-to-peer interaction using CRDTs.
\system{} is inspired by the above designs; additionally, \system{}
supports collaboration groups and peer groups, and ensures that
migration is seamless.

\section{Conclusion}
In this chapter we first analyzed existing approaches for automatically learning
API-usage updates from examples and then studied other related approaches in the domain,
which were either learning semi-automatically, or learned different kinds of transformations.

When we discussed the methods used by automatic approaches to infer
API usage updates, we identified for each approach its inference
capabilities in terms of the taxonomy.
We found that only a few approaches can infer transformations from
examples containing noise or transformations with multiple variants,
and only one can infer transformations relying on control-flow dependencies.
In order to handle Linux kernel API usage updates, we need to design an approach
that specifically targets these challenges.
