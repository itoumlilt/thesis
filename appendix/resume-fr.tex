% !TEX root = ../main.tex
%
\chapter{Résumé}
\label{appendix:resume-fr}

La collaboration à large échelle sur Internet est un domaine d'application 
en pleine expansion,
comme peuvent en témoigner les éditeurs de texte partagés tels que Google Docs
ou Microsoft Office 365; 
les systèmes de partage de fichiers comme Dropbox ou NextCloud.
Certains appareils dotés de capacités de réalité augmentée prennent en charge des 
jeux géolocalisés tels que Pokémon Go ou Harry Potter Unite,
ou des applications collaboratives de modélisation et de conception d'objets 
en 3D.

Les systèmes existants sont basés sur des plateformes de stockage dans le nuage
(\textit{Cloud}),
rajoutant parfois une mise en cache ad hoc au niveau de l'application.
Les violations de cohérence sont courantes dans ce type de configurations,
ce qui perturbe les utilisateurs et rend la conception des applications plus 
contraignante pour les développeurs.
De plus, la prise en charge du fonctionnement en déconnecté est limitée,
voire inexistante.
En effet, les utilisateurs interagissent uniquement par le biais du 
\textit{Cloud},
même lorsqu'une communication directe serait possible entre eux,
puisque ces systèmes sont dépourvus de fonctions de collaboration telles que 
la gestion des groupes et des versions.

Cette thèse explore ces problèmes en profondeur,
en étudiant l'état de l'art lié aux systèmes de stockage 
et services de synchronisation des données aux bords du réseau,
et présente la base de données et l'intergiciel \textit{Colony},
conçu pour répondre aux problématiques exposées.
L'une des principales exigences est une approche \textit{Edge-First}
qui stocke les données au plus près de l'utilisateur, 
directement sur son dispositif,
afin de fournir une disponibilité,
une réponse rapide et transparente indépendamment de la connectivité du réseau
et de l'emplacement géographique,
et afin que les utilisateurs soient propriétaires de leurs propres données.
Cependant, cela rend difficile la satisfaction des attentes en matière
de cohérence et de fraîcheur des données.

Pour répondre à ces défis,
nous avons fait le choix d'adopter une approche hybride,
en fournissant les plus fortes garanties de cohérence compatibles avec la 
disponibilité (modèle \textit{TCC+} formalisé dans la contribution),
tout en renforçant davantage (avec un modèle \textit{SI} présenté 
dans l'état de l'art) ces garanties dans les zones les plus proches 
et connectées (que nous appelons \textit{zones à cohérence forte}).
De plus, nous utilisons le format de données \textit{CRDT} afin d'assurer 
les propriétés de convergence sans avoir recours à un retour en arrière
(\textit{Rollback}).
Un défi connexe est le surcout des métadonnées liées à la gestion de la 
concurrence (gros vecteurs d'horloges),
que nous avons limité grâce à une topologie en arbre flexible,
couplée aux zones à cohérence forte.

Pour répondre aux exigences de la collaboration de groupe,
notre solution versionne les données,
permet à un groupe périphérique 
(terminaux d'utilisateurs géographiquement proches et connectés)
de synchroniser leurs versions sans dépendre du serveur central,
tout en fournissant des garanties de sécurité collaborative.
Notre conception fournit un accès uniforme aux données sur un spectre allant
du nuage central (\textit{Cloud}) à la périphérie du réseau (\textit{Edge}).

Enfin, cette thèse aborde un certain nombre de défis de conception et
de mise en oeuvre du système,
notamment le fonctionnement déconnecté,
le consensus à ordre total à la périphérie du réseau
et l'absence de points de défaillance uniques malgré la topologie en arbre.

Les contributions de cette thèse peuvent être résumées comme suit:
\begin{itemize}
    \item Une architecture de base de données décentralisée,
    conçue pour les applications collaboratives,
    qui fournit un continuum allant du nuage central à la périphérie du réseau;
    \item Un modèle de cohérence hybride,
    basé sur des garanties causales et transactionnelles à large échelle,
    renforcé à la cohérence d'ordre total dans les groupes de collaboration 
    à proximité géographique au bord du réseau;
    \item Une conception évolutive des métadonnées et de la topologie qui 
    limite l'encombrement des métadonnées de cohérence requises,
    tout en prenant en charge la déconnexion ou la migration transparente 
    d'un noeud ou d'un groupe entier;
    \item Une nouvelle approche du contrôle d'accès qui s'appuie sur 
    le modèle de cohérence présenté;
    \item La conception et l'implémentation efficaces de \textit{Colony},
    et son évaluation expérimentale démontrant les avantages de notre approche.
\end{itemize}

Notre évaluation expérimentale montre que : 
la mise en cache locale et de groupe améliore le débit de 1,4× et 1,6× 
respectivement, 
et le temps de réponse de 8× et 20×, 
par rapport à une configuration classique de nuage ; 
la performance en mode déconnecté reste la même qu'en mode connecté ; 
la transition et la migration sont transparentes dans les deux modes de 
connexion.
