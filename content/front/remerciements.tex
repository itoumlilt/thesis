% !TEX root = ../main.tex
%
\pdfbookmark[0]{Remerciements}{Remerciements}
\addchap*{Remerciements}


\textit{"Il va falloir que tu apportes ta tasse de café"},
m'avaient conseillé \bfemph{Maxime} et \bfemph{Florian},
pendant que ce dernier libérait son bureau pour me léguer
ce qui allait devenir mon espace de travail durant les 5 années qui suivirent,
et m'invitèrent ensuite à ma toute première \textit{pause café}.
Je compris très vite l'importance de ces moments d'échange entre collègues,
mais j'étais loin de me douter du soutien capital que ces discussions allaient
m'apporter sur le plan scientifique comme émotionnel,
loin de réaliser qu'une pause café en conférence pouvait dessiner 
ma contribution principale, nouer des collaborations,
et à quel point la communication est un pilier fondamental à l'effort 
collectif qu'est la recherche,
loin de penser que j'allais devenir fournisseur officiel de ce breuvage psychotrope
dans mon équipe.
C'est donc tout naturellement que je souhaite remercier ici toutes les personnes qui
ont accompagné ce long périple que fut ma thèse, par leur soutien, 
leurs encouragements ou leur jus de neurones.

Tout d'abord, 
je tiens à remercier mes rapporteurs,
\bfemph{Sébastien Monnet} et \bfemph{Étienne Rivière},
pour le temps qu'ils ont consacré à la lecture attentive et à l'évaluation
de ce rapport.
Merci également à tous les autres membres du jury,
\bfemph{Sonia Ben Mokhtar}, \bfemph{Annette Bienuisa}, \bfemph{Valérie Issarny}, 
\bfemph{Pierre Sutra} et \bfemph{Marek Zawirski}.

Ensuite, 
cette thèse n'aurait pas été possible sans la supervision de 
\bfemph{Marc Shapiro}. 
Marc, 
merci à toi pour ta confiance, 
de m'avoir accompagné et suivi toutes ces années, 
d'avoir été suffisamment strict pour orienter mes idées,
mais suffisamment souple pour me laisser maîtriser ma thèse,
ton expertise en matière de rédaction de papiers de recherche
a rendu cette thèse plus facile qu'elle n'aurait pu l'être.
Malgré tes occupations de plus en plus nombreuses,
tu as toujours été disponible quand j'en avais besoin,
merci pour ces innombrables \emph{meetings},
parfois même à des heures improbables.
Je ne serai jamais assez reconnaissant pour tout ce que j'ai appris à tes côtés,
mais je m'efforcerai de garder la rigueur que tu m'as enseignée pour faire de la 
recherche en systèmes,
et j'espère que je serai à la hauteur de tes attentes.

J'aimerais également adresser un remerciement très particulier à \bfemph{Pierre Sutra},
qui n'était pas officiellement mon encadrant,
mais qui a suivi mes travaux avec assez de recul pour m'apporter le bon sens 
nécessaire à leur publication.
Son esprit critique et sa capacité à proposer des pistes de recherche 
prometteuses furent les bienvenues à de nombreuses reprises.

Je souhaite ensuite remercier les membres du \bfemph{LIP6},
en particulier les permanents de l'équipe \bfemph{Delys},
d'abord en tant qu'enseignants ayant contribué à mon intérêt pour les 
systèmes,
puis en tant que collègues auprès de qui j'ai eu la chance de travailler.
Merci \bfemph{Pierre Sens},
pour tes nombreux conseils,
pour nos échanges toujours source d'encouragement et d'optimisme.
Merci \bfemph{Jonathan Le Jeune},
d'être le reflet de la convivialité du Nord.
Merci \bfemph{Julien Sopena},
pour ton dévouement en tant qu'enseignant,
ta confiance en mon enseignement,
tes précieux conseils,
et pour l'étendue de ta curiosité et ton esprit critique qui ont souvent
enrichi nos discussions,
tant sur le plan scientifique qu'humain.

Viennent ensuite les camarades doctorants des équipes Delys, MoVe et Whisper.
En commençant par les plus anciens, ceux qui ont facilité mon intégration.
Merci \bfemph{Antoine Blin}, mon premier mentor et notre expert du temps-réel;
\bfemph{Gauthier Voron}, qui apprécie autant les hyperviseurs que la beauté de Perl;
\bfemph{Damien Carver}, spécialiste des conteneurs, des tableaux et des sports de glisse;
\bfemph{Hakan Metin}, qui sait tout expliquer par 1-2-3-4;
\bfemph{Florent Coriat}, notre expert et rescapé réseau;
Ainsi qu'à toute la \emph{génération Regal}: \bfemph{Maxime Lorrillere, Rudyar Cortés, Lyes Hamidouche, Alejandro Tomsic}.
Ensuite je souhaite remercier celles et ceux qui ont partagé mon quotidien au labo,
en commençant par mon voisin de bureau, 
\bfemph{Francis Laniel}, le plus sage des Stéphanois;
\bfemph{Saalik Hatia}, ses 404 cellulaires portatifs et ses Tech Tips;
\bfemph{Jonathan Sid-Otmane}, sa grande curiosité tech, sauf pour la 5G;
\bfemph{Vincent Vallade}, champion du monde du SAT;
\bfemph{Lucas Serrano}, Légion Döner du python;
\bfemph{Arnaud Favier}, premier doctorant système à promouvoir le web dev;
\bfemph{Laurent Prosperi}, seul camarade comprenant l'intérêt de la \emph{VO2 Max};
\bfemph{Ludovic Le Frioux}, amateur du Clash of Code;
Mais également à toute la génération \bfemph{Dimitrios Vasilas, Gabriel Le Bouder, Sreeja Nair, Daniel Wilhelm, Benoît Martin, Sara Hamouda, Guillaume Fraysse, Marjorie Bournat, Élise Jeanneau, Mathieu Lehaut, Cédric Courtaud, Darius Mercadier, Yoann Ghigoff} avec qui j'ai partagé maintes conversations passionnantes.
J'aimerais également souhaiter plein de courage à la \emph{génération Adum},
\bfemph{Célia Mahamdi, Aymeric Agon-Rambosson} et \bfemph{Étienne Le Louët}.

Je souhaite remercier tout particulièrement,
\bfemph{Maxime Bittan} et \bfemph{Redha Gouicem},
que je suis heureux de compter parmi mes amis depuis les tout premiers cours
à l'université et jusqu'au laboratoire,
merci pour les très bons moments (d'informatique, comme de rires) passés ensemble.

Ma vie à l'université n'aurait pas été la même sans les bons moments passés 
au \bfemph{DAPS}, à l'\bfemph{AS}, à l'\bfemph{AEIP6/ALIAS} et toutes les activités 
proposées par l'écosystème associatif de \bfemph{Sorbonne Université}.
L'université m'a également permis de coupler mes études à ma pratique sportive
qui m'a souvent aidé à tenir mentalement durant cette thèse,
j'en profite pour remercier mon club d'athlétisme, l'\bfemph{AO Charenton}, et \bfemph{adidas}.
Le lien fort entre co-équipiers, la discipline à toute épreuve et la persévérance,
sont des valeurs fondamentales qu'on retrouve en athlétisme comme en doctorat.

Ces années de thèse auraient été infiniment moins agréables et enrichissantes sans mes amis,
de \bfemph{Gaillon} à \bfemph{Paris}, du \bfemph{Kop Sud} au \bfemph{Central}. Merci les copains.

Et enfin, je tiens à remercier \bfemph{ma famille} pour son soutien indéfectible
depuis toujours.
Maman, Papa, les mots ne suffiraient pas pour éprouver toute ma reconnaissance pour tout ce que vous avez fait pour moi.
Grâce à vous, j'ai pu étudier ce domaine qui me passionne et m'investir dans cette 
aventure de longue haleine.
Grâce à vous, je suis devenu l'homme que je suis aujourd'hui. 
Merci pour votre confiance et le confort que vous m'avez offert pour mener à bout mon projet d'études.
J'espère que vous êtes aujourd'hui fiers des résultats de mon parcours.

Ces remerciements ont été bien plus longs et dénués de blagues que ce que j'aurais espéré,
mais croyez-moi,
j'ai éprouvé bien plus d'émotions et de nostalgie que je n'ai pu le résumer ici,
et si par mégarde j'ai oublié quelqu'un, veuillez m'en excuser.

Merci encore, à toustes, même celleux qui ne consomment pas de caféine.
