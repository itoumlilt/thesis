% !TEX root = ../main.tex
%
\pdfbookmark[0]{Résumé}{Résumé}
\addchap*{Résumé}
\label{chap:résumé}

\vspace*{8mm}
La distribution et la réplication des données en périphérie du réseau apportent une réponse immédiate, une autonomie et une disponibilité aux applications de périphérie, telles que les jeux, l'ingénierie coopérative ou le partage d'informations sur le terrain. Cependant, les développeurs d'applications et les utilisateurs exigent les meilleures garanties de cohérence possibles, ainsi qu'un support spécifique pour la collaboration de groupe. Pour relever ce défi, Colony garantit la cohérence Transactional Causal Plus Consistency (TCC+) à échelle planétaire, en complément de l'isolation des instantanés au sein des groupes de périphérie. Pour favoriser le passage à l'échelle, la tolérance aux pannes et la sécurité, sa topologie de communication logique est arborescente, avec des racines répliquées dans le nuage principal, mais avec la possibilité de migrer un nœud ou un groupe. Malgré cette approche hybride, les applications bénéficient de la même sémantique partout dans la topologie. Nos expériences montrent que la mise en cache locale et les groupes collaboratifs améliorent considérablement le débit et le temps de réponse, que les performances ne sont pas affectées en mode hors ligne et que la migration est transparente.

\textbf{Mots-clés:} 
Cohérence Causale, Systèmes Collaboratifs, Informatique au bord du réseau,
Géo-Réplication, Systèmes Pair-à-Pair