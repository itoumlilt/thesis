% !TEX root = ../../main.tex

\system{} ensures convergence by using operation-based CRDTs, which
merge concurrent conflicting operations deterministically
\cite{syn:rep:sh143}.
Similarly to recent CC designs \cite{db:syn:1808, rep:pro:sh182},
\system{} separates (internal) state management from (external)
visibility: the backend layer transmits and stores the state efficiently,
without regard for correctness, whereas the visibility layer manages
metadata and ensures that an application observes only those states that
satisfy the TCC+ guarantees.

\section{Versioning}
\label{sec:versioning}

\system{} stores an object persistently as a base version and
a journal of updates since the base version.
To materialize an arbitrary object version, the cache first reads the
base version from the store, and applies the missing updates from the
journal.
Occasionally, the system advances the base version.

A transaction reads from its snapshot, logs its updates to the
journal, and materializes new versions in a private buffer.
When the transaction commits, it updates the cache from the buffer.
Both the updates recorded in the journal, and object versions that
result from committed transaction $T$, are labeled with vector $T.C$
and dot $T.D$.

\section{Edge caching}
\label{sec:edge-caching}

An edge node cannot replicate the whole database, but can only cache
some small fraction of it.
An edge client may declare \emph{interest} in some object to add it to
its node's cache.
The connected DC regularly informs the client of updates to its interest
set.

At any point in time, the state vector of an edge node is the LUB of the
state received from its connected DC (itself $\le$ the DC's current
state vector) and the commit vectors of local transactions.
Choosing a snapshot vector $\le$ the node's state vector ensures that
every read could be satisfied either from the local cache or from the
connected DC.
It may happen that the client requires an object version that cannot be
retrieved (in the cache, from the DC or from another node), in which case
the transaction cannot proceed.
This limitation of availability is inherent to the edge environment.
