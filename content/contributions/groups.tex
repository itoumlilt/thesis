% !TEX root = ../../main.tex

\system{} supports two distinct group mechanisms:
\begin{inparablank}
\item
  the peer group, an SI zone at the edge, and
\item
  the collaboration group, nodes that update the same data.
\end{inparablank}
Peer groups are disjoint, whereas collaboration groups may overlap.
All nodes in a peer group are in the same collaboration group.

\section{Peer groups}
\label{sec:peer-groups}

A peer group is a set of nodes with high-availability, low-latency
connection to one another.

It makes sense to provide SI within the group.
This enhances the user experience, and simplifies metadata management.
A peer group creates opportunities to improve performance, by pooling
resources into a collaborative cache, and to decrease network load to
the cloud by collecting the updates from many clients.

It may also isolate itself from the cloud, keeping out distracting
remote updates and improving security.

In summary, a read by a member of the group is satisfied either from its
local cache, or from that of a neighbor, or from the connected DC\@.
Reads and writes satisfy the TCC+ and SI guarantees, as well as security
guarantees described elsewhere.

Conceptually, a peer group consists of four related components, with
distinct roles:
\begin{inparablank}
\item
  managing group membership,
\item
  sharing content within the group,
\item
  communicating with the outside,
and \item
  enforcing the SI order.
\end{inparablank}
They are described in further detail below.

\subsection{Membership}
\label{sec:groups-membership}

Membership of a peer group is seeded and managed by a single node,
called the group's \emph{parent}.
The parent maintains a connection to each of the group member, stores
the list of the members, and informs them of any membership change.
The parent is fixed but arbitrary, possibly located in the DC or on a
point-of-presence (PoP) server.
A node may serve as a member and a parent at the same time.

To join or leave a group, a node contacts the group's parent.
The parent responds with the membership list, as well as the session
security key (described shortly).
When a node migrates between groups, it uses the migration protocol
previously described (Section~\ref{sec:device-migration}); the new group
must be causally compatible with the node's state.

\subsection{Content Sharing}
\label{sec:content-sharing}

Using the membership list, the group members and their parent maintain
point-to-point connections.
Above these connections, they construct a collaborative cache using a simple peer-to-peer
protocol.
Each member publishes its current interest set to all its neighbors
(other members and parent).
This subscribes the member to receive all updates to its interest set.
When a member updates an object in a neighbor's interest set, it pushes
that update in a best-effort manner.
Conversely, if a member observes that it is missing an update to its
interest set (by examining the visibility log described below), it pulls
the transaction from some neighbor.
Objects evicted from a cache are unsubscribed to save resources.

When a node first joins a group, or when it extends its interest set,
some neighbor will push the objects it requires, as described above.

The parent maintains an interest set that is the union of those of the
group members.
It subscribes for updates outside the peer group on behalf of its
members, as detailed next.

\subsection{Communicating Outside the Group}
\label{sec:comm-outs-group}

As illustrated in Figure~\ref{fig:topology}, a subtree communicates with
another one via some common ancestor.
For simplicity, the following description assumes this ancestor is
its connected DC\@.%


\footnote{

So far, we glossed over the fact that a DC contains a number of
concurrent shard servers. The group might communicate with any of
these servers. }

Let us call \emph{synchronization point} (sync point) a node within a
group that communicates with the DC\@.
In the common case, this is the parent, but any member may also
unilaterally become a sync point (for instance before migrating a
transaction to the DC, Section~\ref{sec:trans-migr}), thus avoiding any
single point of failure.

A sync point sends all the updates that the DC is missing, and
symmetrically subscribes to updates to its interest set.
Importantly, the sync point makes updates visible to the DC in the
visibility order described in the next section.
This ensures that different sync points send identical information.

\subsection{Transaction Protocol for Peer Groups}
\label{sec:group-txn}

A peer group as a whole should behave like a single, sequential edge
node, from the perspective of the rest of the system.
To ensure sequential ordering, causality and progress within the group,
\system{} relies on EPaxos \cite{syn:rep:1843} within the peer group.
Compared to other consensus protocols, EPaxos improves availability and
performance, by allowing any group member to become the leader for some
particular transaction, and by minimizing synchronization between
non-conflicting transactions.

In addition to improving the user experience, consensus is essential to
correct metadata management.

Recall from Section~\ref{sec:comm-outs-group} that possibly multiple
sync points send transactions to the DC\@.
Without consensus, conflicting transactions would be sent in different orders,
breaking causality and causing unsafe commit vectors.

When a peer node commits a transaction, it submits it to EPaxos.
EPaxos ensures consensus on the order in which versions become visible
sequentially 
according to SI, which we call \emph{visibility order.}
Every peer maintains the list of visible transactions in a
\emph{visibility log.}

The transaction then executes in isolation against the local cache.
Its dependencies are the union of the state vector, the node's previous
transactions, and the transactions in the node's visibility log.

The node assigns it a unique dot, executes and persists its updates in
a local journal, and updates its local cache.

Within a peer group, two different variants of commit exist.
In the first, the node submits the transaction to EPaxos in the critical
path of commit.
This has the effect of ordering the commitment of conflicting
transactions within the peer group, possibly leading to aborts;
non-conflicting transactions commit in parallel.
This variant maintains Parallel Snapshot Isolation (PSI) within a group
\cite{rep:syn:1661}, ensuring that the group behaves as an SI zone.

The second variant follows a similar approach to
Section~\ref{sec:edge-trans-prot}.
It assumes that transactions never conflict.
The transaction commits locally as soon as it reaches the commit
statement, and a new transaction can follow immediately.
The transaction is then submitted to EPaxos in the background.

Committed transactions become visible in the order
assigned by EPaxos.
A sync point sends visible transactions to the connected DC according
to the visibility order, where they get assigned a commit timestamp.

\section{Migration Between Peer Groups}
\label{sec:migration-to-peer-group}

Just as a node can migrate between DCs
(Section~\ref{sec:device-migration}), it may migrate between peer
groups.
Similar consistency issues occur here.
In this case, the base version of cached objects on the migrating node
must be compatible with that of the new group.
If the client is not missing any dependencies, or can retrieve them,
then migration is seamless.

However, if the client is missing dependencies and the new peer group is
offline, migration cannot succeed.
If the client waits, its pending commits remain logged until the
communication problem is fixed and they can be merged into the DC\@.
In the meantime, the client might start a session with the new group,
but its pending updates in the old session become invisible.
Alternatively, the client might attempt to  migrate again.

\section{Collaboration groups}
\label{sec:security-protocol}

The mechanisms related to collaboration groups are trust management and
versioning.

Messages are protected using symmetric cryptography.
The authentication service  provides a client with a session key
per shared object, which she uses to decrypt data and sign her updates.
This ensures that only legitimate clients can read an object.
The key remains valid through disconnection and reconnection.

To keep out untrusted or unwanted updates, we leverage the separation
between state and visibility, previously discussed in
Section~\ref{ch:protocol-design}.
Recall that an update is visible only if it satisfies the TCC+
invariants.
In addition, it is visible only if it satisfies collaboration constraints.

To manage trust, the security administrator sets ACL.
Furthermore, a collaboration group can, for instance, restrict
visibility to include only versions produced within the group.
An update that does not satisfy the corresponding ACL or group
constraints remains invisible, and transitively the updates that depend
upon it.
Thus, security policies and groups can evolve dynamically.
Technically, this violates the monotonicity invariant, but in a very
restricted manner.
The store remains TCC+, but security and group constraints expose only a
variable-size window thereof.

The data model of \system{} together with \tccp is able to express 
(and maintain) 
several classes of applicative invariants.
We illustrate this below using two examples, a control access system and a 
collaborative social application.

We consider a fail-stop distributed model of computation.
Clients execute wait-free operations in \system{} whatever failures occur in the 
system.
