% !TEX root = ../../main.tex

The \system{} middleware is designed to provide a simple API for developing and deploying collaborative applications.
This section presents its implementation and programming interface.
The code is open-source and available on GitHub.%


\section{Modular design}
\label{sec:modular-design}

\system{} enjoys a modular design, where each component (DC,
PoP, Client, API) can be deployed and run independently.

Server components are implemented in Erlang/OTP, a functional language
built for distribution and concurrency. We provide Docker containers
for each server module.

The client component has two versions: a lightweight Typescript
library for web applications, and an Erlang module that can run on the
edge device, serving multiple applications through a language-agnostic
Protocol Buffer interface.

\section{API and programming model}
  
An application node connects to a session manager (currently implemented
in  the core cloud), which authenticates the node.
With the session opened, the node may join a collaboration or peer group,
and run transactions accessing database objects.
The node is notified of group change events (e.g., a new peer joins).

The database stores CRDT objects, such as counters, registers, sets,
maps, or sequence datatypes.

\system{} can also support custom data types, but details are omitted.

An object is stored in a namespace called a bucket.
Opening a bucket caches it in the node; optional parameters can specify
cache policies (e.g., LRU, writeback, etc.).
The application can subscribe to an object's update events, in order to
implement reactive programming patterns.
A transaction is atomic (all-or-nothing) against multiple updates, and
reads a  TCC+-consistent snapshot of its opened buckets.
\system{} supports both interactive and batch transactions.

\begin{figure}[th]
  \lstinputlisting[style=typescript,linerange={8-25}, xleftmargin=3em]{docs/typescript/src/api.ts}
  \caption{
    \labfigure{api}
    An example of \system{} program.
  }
\end{figure}

The TypeScript example in \reffigure{api} illustrates the API.
This application opens a session (Line~1).
Then, it creates and increments a CRDT counter object (Lines~2--5).
Then it connects to a peer group (line~7), and updates the grow-only map
(gmap) \code{{\color{RoyalPurple}{``myMap''}}} in a transaction
(lines~8--12).
This map contains references to a register object (key
\code{{\color{RoyalPurple}{``a''}}}) and a set object (key
\code{{\color{RoyalPurple}{``e''}}}).
The counter update and the commit are both asynchronous 
(Lines~9 and~13), returning a promise.
At line~13, the client waits for the promise, and displays the content
of the set.

\section{Communication protocol}

Edge nodes communicate over WebRTC\@.
Opening a client session occurs in the signaling phase of WebRTC and
currently relies on a server in the core cloud, to simplify
authentication and trust management.
The session provides the networking information required to communicate
with the system, i.e.,
\begin{inparablank}
  \item the IP addresses and ports  of nearby peers, and
  \item the keys required to establish secure point-to-point connections
    with them.
\end{inparablank}%
\footnote{
%
  The first connection is established via STUN.
  If this fails (due to a firewall or NAT), \system{} falls back to
  using TURN \cite{wing2008session}.
}
To migrate to a different peer group, the node relies again on the
session server.


\section{Storage}
\label{sec:object-store}

Cloud nodes (DCs and PoPs) have secondary storage and persist their data
to it.
They also cache data in memory for performance.
Data in a DC is sharded by consistent hashing across multiple server
machines, leveraging \texttt{riak\_core} \cite{riak-core}.

We do not assume that a far-edge node has disk, and store their data in
browser memory.
When a disconnected client reconnects again, it repopulates its cache,
either from its peer group's content-sharing network, or from its
connected DC\@.

\section{Security}

The authentication keys received from the session server serve to
encrypt communication between nodes, using symmetric encryption.
This ensures that only authenticated clients are able to observe and
update objects.
Decentralised authentification \cite{sec:rep:1823} is left for future
work.

A system administrator can set a security policy with the help of access-control lists (ACLs).
An ACL is a tuple from the set $\mathsf{objects} \times \mathsf{users} \times \mathsf{permissions}$.
It defines that a given user is granted access to some object and the operation she is allowed to execute on that object.
Right inheritance (RI) is modelled using two forests, atop $\mathsf{objects}$ and $\mathsf{users}$.
%
If user $u$ inherits from user $v$, then $u$ holds the same ACL as $v$.
Similarly, if an object $x$ inherits from some object $y$, then any ACL granted on $y$ also holds for $x$.
Checking an ACL evaluates a first-order logic predicate over the RI and ACL relations following the above logic.
For instance,
\begin{inparaitem}
\item[(C1)] $(\mathsf{book}, \mathsf{Alice}, \mathsf{own}) \in ACL$, or the more complex\\
\item[(C2)] $(\mathsf{book}, \mathsf{shelf}) \in RI \land (\mathsf{shelf}, \mathsf{Bob}, \mathsf{read}) \in ACL$
\end{inparaitem}
specify respectively that $\mathsf{Alice}$ owns a $\mathsf{book}$ and that this book is on a $\mathsf{shelf}$ readable by $\mathsf{Bob}$.

An ACL check must respect the order in which clients modify both data
and the security policy, to avoid unexpected behavior.
More precisely, the system must ensure \cite{sec:rep:1842} that:
\begin{inparaenum}[\textit (i)]
\item
  ACLs are applied in the order they were issued, and
\item
  ACL checks are evaluated on a fresh copy of data and metadata.
\end{inparaenum}
If data and security metadata are mutually consistent according to
\tccp, the first constraint is trivially satisfied.

Formalizing the second one, if $x$ and $y$ are related by some ACL
check, and if $ux$ occurs in real time before $vy$, then every
transaction that reads $vy$ must access at least version $ux$. This is
a form of external consistency, and is therefore not available under
partition \cite{rep:pan:1628}.

Let us use an example to illustrate the problem with the second
constraint.
Consider predicate C2, and assume that Alice, Bob and Carl share the
bookshelf.
Suppose Alice removes a book from the shelf on  her node, while
Bob makes the shelf readable by everyone.
The two are concurrent from the causality perspective, and thus Carl may
observe them in any order.
However, by the second constraint, if Bob's update occurs later in real
time, then Carl must never see Alice's book on the shelf.
If Bob's node is disconnected or slow to transmit, this requirement is
violated.

\system{} alleviates this problem as follows.
First, object versions are visible according to the local copy of the
$ACL$ and $RI$ relations.
Second, it defers ACL checks to after commit.
A committed transaction that fails an ACL check is not visible.
In the above example, Alice's book may appear briefly on Carl's node;
but as soon as Bob's update is delivered, it will disappear.
