% !TEX root = ../main.tex
%
% \chapter{Conclusion}
% \label{ch:conclusion}

Many web applications are built around direct interactions among users, 
from collaborative applications and social networks to multiuser games. 
Despite being user-centric, 
these applications are usually supported by services running on servers that 
mediate all interactions among clients. 
When users are in close vicinity of each other, 
relying on a centralized infrastructure for mediating user interactions 
leads to unnecessarily high latency while hampering fault-tolerance and scalability.

This thesis first explored the edge computing paradigms and the challenges 
that are inherent to its network and architecture.
Then, we presented the safety, scalability, security, hierarchy and programmability
requirements for todays collaborative edge applications.
We also explored the solutions from the state of the art.

In the second part of the thesis, 
we presented the design of \system{}, a system that brings
geo-replication guarantees to the Edge.
\system{} allows applications to run transactions in the client machine,
for common operations that access a limited set of objects, with
immediate, consistent and offline response, or in the DC, for
transactions that require accessing a large number of objects.
\system{} also proposes a client-assisted failover mechanism that trades
response time by a small increase in staleness.

\section{Limitations and perspectives}

Several aspects remain open for improvements and investigation.

\paragraph{Better caching heuristics} 
the current design caches at the client side the least recently used versions
of clients objects.
We would like to take into account the interest set of the application,
and have more level of LRU caches,
to avoid evicting objects that are actively used by the application
while loading temporary objects.

\paragraph{Support for transaction migration}
Client computation resources can be poor and limited, although
being resource-friendly and metadata lightweight, in \system{}, some heavy operation can
be done faster using the Datacenter ressources. 
We have explored a hybrid model where we
can move computation from the client to the server in the heavy jobs case. This raises
some interesting challenges like preserving the causal state of the client, handling updates
and scheduling operations.

\paragraph{Data and Point-of-Presence placement}
Placing clients at different levels of the hierarchy, in particular in
Content Delivery Network points of presence, might improve perceived
response time even more.
In addition to prefetching heuristics and better caching at the 
Point-of-Presence level.

Extending the peer-to-peer communications across edge services would also make 
the system less dependent on the cloud for updates dissemination.
