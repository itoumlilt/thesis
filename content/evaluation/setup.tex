% !TEX root = ../../main.tex

We deploy each \system{} component (edge client, cloud server, peer group, etc.)
as a Docker container, on a set of dedicated servers in a cluster.
Each server has two Intel Xeon Gold CPUs, each with 16~cores per CPU,
128\,GB of memory, and 2~TB of (spinning) hard disk.
Nodes are all connected through 10\,Gb/s network switches.
A monitoring server, deployed on a separate container, captures the performance 
metrics.

We measure an average 0.15\,ms network latency within the cluster.

We use the Linux traffic shapping tool (\texttt{tc}) to simulate larger network 
latency, with a mean of
50\,ms for mobile cellular data and 10\,ms for carrier Ethernet.
DCs are connected in a mesh using RabbitMQ sockets above TCP;
peer groups are connected using WebRTC.

The performance evaluation in this chapter is performed using the \system{}Chat 
Benchmark application presented in Chapter~\ref{ch:eval:benchmark-app},
as this scenario has the largest set of collaborative transactions on 
shared data and objects between users.

To compare performance, we first evaluated the performance of \system{} compared
to classical cloud and single client caching (without peer-to-peer groups), 
to demonstrate the benefits of our approach; we then ported the \system{}Chat
application on top of representative platforms for web-based data synchronization:
Yjs \cite{yjsyjsSh56:online} 
(a widely used open source technology, with 5.5k Github stars) 
and Automerge \cite{kleppmann2018automerge},
both are using state-based CRDTs, Automerge being the JSON based implementation.
