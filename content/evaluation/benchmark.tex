% !TEX root = ../../main.tex

\paragraph{Overview}

\system{}Chat emulates a team collaboration application modeled after the
Slack and Mattermost communication platforms \cite{slack,mattermost}.
It consists of approximately 1500 lines of Typescript.
\system{}Chat represents its three main entities, users, 
workspaces and bots with the help of CRDT objects.
%
In detail, 
a \textit{user} has a profile, 
a list of events, 
a set of friends, 
and a set of workspaces she is a member of.
A \textit{workspace} contains the users that collaborate through the application
 and a set of channels.
It also maintains the status of the users within the workspace 
(e.g., owner, ordinary, invited, or deleted).
A \textit{channel} holds a brief description, and the list of messages
posted to it.
%
A \textit{bot} is a special kind of user.
It automatically triggers an action when it observes some event, 
or a specific message on a channel.
For instance, 
a bot might monitor activities within a file-system tree, 
or display weather information.
Bots play an important role in the benchmark, 
as they generate a large number of update transactions.

The TCC+ guarantees of \system{} ensure that there are no ordering anomalies in 
the application.
For instance, 
an answer is guaranteed to be visible in a chat after the corresponding question.
Moreover, 
transactions are atomic, 
allowing maintaining invariants such as ``a user is in a workspace if and only 
if the workspace is in the user's profile.''
Furthermore, within an SI zone such as a peer group, 
users observe updates in the same order, 
greatly simplifying collaboration.

\paragraph{Workload}
The workload consists of a modified trace from a popular Mattermost
server.
The trace contains the actions of around 2,000 users spreads over 3 workspaces; 
each workspace holds 20 channels on average.
A user can be in more than one workspace, and one of the workspaces contains 
1,000 users.
%
Around 10\% of the users are bots that act randomly upon receiving a message on 
the channel, they have subscribed to.
An action of a user follows a 90/10 read/write ratio.
A user refreshes its local copy of a channel every 5 transactions.
%
The trace follows a Pareto distribution for the action, where 20\% of the users 
execute 80\% of operations.
It contains 40 days of activity in total on the Mattermost server and exhibits 
a diurnal cycle.
In the experiments, the trace is accelerated to execute in a few minutes only.

For each experiment, 
we indicate when users are scattered in peer groups, 
or directly connected to a remote DC.
%
The experiments use the second variant of
the peer group commit protocol, i.e., EPaxos is off the critical path of
commitment (Section~\ref{sec:group-txn}).
%
The current version of our benchmark does not exercise transaction migration 
(Section~\ref{sec:trans-migr});
this will be added in future work.
%
Each experiment is executed 10 times, 
and we report the average.
