% !TEX root = ../../main.tex

This evaluation part of the thesis presented an experimental evaluation 
demonstrating the benefits of our approach.

Our experimental evaluation shows that: 
local and group caching improve throughput by 1.4× and 1.6× respectively, 
and response time by 8× and 20×, compared to a classical cloud configuration; 
performance in offline mode remains the same as online; 
both the offline/online transition and migration are seamless.

\system{} achieves a similar performance to other collaborative edge systems,
even with consensus at the peer groups,
which helps reducing the metadata cost. 
The forest topology and the ability to synchronise from multiple point of failure
makes the resynchronization after failure in less than 2secs even with large 
group of clients.

All the approaches used in our evaluation uses state-based CRDTs,
an improvement idea can be to make use of Delta-based CRDTs \cite{almeida2015efficient}
to improve the resynchronisation time.
