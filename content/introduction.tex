% !TEX root = ../main.tex
%
\chapter{Introduction}
\label{sec:intro}

Internet-scale collaboration is a growing application
area, as evidenced by games such as Overwatch or Ingress,
shared editors such as Google Docs, Microsoft Office 365 or Apple
iWork, or
file-sharing systems such as Dropbox or Nextcloud.

Mobile devices with Augmented Reality capabilities support
location-aware games such as Pokémon Go and Harry Potter Unite, or
collaborative 3D modeling and manufacturing applications
\cite{wang2019comprehensive, collaborativeARusecase, schafer2021survey}.
Virtual worlds, or the Metaverses \cite{dionisio20133d}, 
constitute a subset of collaborative virtual reality applications.
Metaverses are persistent online computer-generated environments where multiple 
users in remote and close physical locations can interact in real time for the 
purpose of work or play.
Many industries are today building Metaverse applications such as Facebook 
(Meta) \cite{Connect213:online},
Microsoft's metaverse work spaces \cite{Microsof66:online} or 
the NVIDIA Omniverse platform \cite{WhatIsth95:online}, 
and many more industry companies \cite{dionisio20133d}.
Those augmented reality worlds generate massive computation for space recognition 
(e.g., using a smartphone camera to scan relevant objects around the user)
on the edge devices, and expect real-time collaboration.

Existing systems are cloud-based, sometimes providing ad hoc
application-level caching.
Consistency violations are common, baffling users and vexing application
developers
\cite{fic:rep:sh172,app:rep:1835,Google-docs-support-offline-issues,Google-docs-offline-issues-complaints,pokemon-go-object-duplication}.
Support for offline operation is spotty.
Users interact through the cloud only, even when direct communication
would be possible.
These systems lack collaboration features such as group management or
versioning.

Cloud computing is an established paradigm and most collaborative applications 
today, 
including those running on mobile devices, rely on the cloud to exchange data, 
download information or offload resource-intensive computations to run on more 
powerful servers and save battery power. 
Massively multiplayer mobile games are an example of a large-scale collaborative
application requiring low latency between players and generating 
resource-intensive computation, 
especially with the advent of augmented reality,
which often involves image 
processing for face recognition or object detection. 
However, for these applications to be usable, the expected response 
must be under 5 milliseconds, which is difficult to 
guarantee when an edge device has to synchronize via remote data centers.

\section{Overview}
\label{sec:intro:overview}

In the first part of this thesis,
we explore the edge computing paradigms and the challenges that are
inherent to its network and architecture.
Then, we discuss the safety, scalability, security, hierarchy and 
programmability requirements for collaborative 
edge applications.
And finally, we present a comprehensive study of the state of the art,
and what has been done in the literature to fulfill the edge computing 
requirements and collaborative applicative requirements. 

In the second part of the thesis,
we present our main contribution, \system{},
a database and middleware designed to address these issues.
A top requirement is an \emph{edge-first} approach \cite{rep:app:1827}
that locates data on the device, to provide availability, fast and
seamless response independently of network connectivity and of location,
and so that users have ownership of their data.
However, this makes it challenging to satisfy consistency and freshness
expectations.
To answer this challenge, \system{} takes a hybrid approach, and
provides the highest consistency guarantees compatible with availability
(TCC+), strengthened further (to SI) in well-connected zones.
CRDT data types ensure convergence without rollbacks.
A related challenge is the overhead of concurrency metadata
(fat vector clocks), which we limit thanks to a flexible forest
topology and to SI zones.

To address the requirements of group collaboration, \system{}
enables edge collaborators to share data without relying on the cloud,
and supports collaborative security features, including end-to-end
encryption.
Our design provides uniform access to data across a spectrum spanning
core cloud to the far edge.

Finally, this thesis addresses a number of design and implementation
challenges, including disconnected operation, consistency under
migration, total-order consensus at the edge, and avoiding single points of
failure despite the tree topology.

\section{Contributions}
\label{sec:intro:contributions}

The main results of this dissertation are as follows:
\begin{itemize}[leftmargin=*]
\item
  A decentralized database architecture, designed for collaborative
  applications, that provides a continuum spanning from the core cloud
  to the far edge.
\item
  A hybrid consistency model, based on causal and transactional 
  guarantees globally, strengthened to total-order consistency in edge
  collaboration groups and in geographical proximity.
\item
  A scalable metadata and topology design that bounds the overhead of 
  the required consistency metadata, and that supports seamless
  disconnection or migration of a node or of a whole group.
\item
  A novel approach to access control that leverages the consistency
  model.
\item
  Efficient design and implementation of \system{} and
  its experimental evaluation demonstrating the benefits of
  our approach.
\end{itemize}

Our experimental evaluation shows that:
local and group caching improve throughput by 1.4$\times$ and
1.6$\times$ respectively, and response time by 8$\times$ and 20$\times$,
compared to a classical cloud configuration;
performance in offline mode remains the same as online;
both the offline{\slash}online transition and migration are seamless.

\section{Publications}
\label{sec:intro:publications}

Some of the results presented in this thesis have been published as follows:

\begin{itemize}
  \item Ilyas Toumlilt, Pierre Sutra, Marc Shapiro. Highly-Available and Consistent Group Collaboration at the Edge with Colony. Middleware 2021 - 22nd International Middleware Conference, Dec 2021, Québec / Virtual, Canada. ⟨10.1145/3464298.3493405⟩. ⟨hal-03353663v3⟩ \cite{toumlilt:hal-03353663}
  \item Ilyas Toumlilt, Alejandro Tomsic, Marc Shapiro. Vers une cohérence causale évolutive sans chaînes de
  ralentissements. Compas 2018: Conférence d'informatique en Parallélisme, Architecture et Système,
  Jun 2018, Nice Sophia-Antipolis, France. ffhal-01860334f \cite{toumlilt:hal-01860334}
\end{itemize}

During my thesis, I explored other directions and collaborated in several projects that have helped me to get insights on the challenges of providing consistency in geo-replicated and edge systems. These efforts have led me to contribute to the following publications and deliverables:

\begin{itemize}
  \item Ali Shoker, Paulo Sergio Almeida, Carlos Baquero, et al. LightKone Reference Architecture (LiRA). https://www.info.ucl.ac.be/~pvr/LiRAv0.9.pdf. 2020 (cit. on p. 3). \cite{LiRAv09p46:online}
  \item LightKone H2020 Project. D6.1: New concepts for heavy edge computing. 2018 (cit. on p. 4) \cite{D61Newco32:online}
  \item LightKone H2020 Project. D6.2: New concepts for heavy edge computing. 2019. \cite{D62Newco60:online}
\end{itemize}

\section{Organization}
\label{sec:intro:organisation}

This thesis is divided into three parts. 
The rest of this document is organized as follows:
\begin{itemize}
  \item Part~\ref{part:background} introduces the common background of our work,
  formulate the problem, presents the existing solutions and discusses the use-case
  requirements, this part is divided into three chapters:
  \begin{itemize}
    \item Chapter~\ref{ch:edge-computing} provides a complete and up-to-date 
    review of edge-oriented computing systems by aggregating relevant challenges 
    on their architecture features, management approaches, and design goals.
    \item Chapter~\ref{ch:requirements} presents a use-case point of view of the
    previous review, discussing the safety, scalability, security, hierarchy and 
    programmability requirements that need to be guaranteed for collaborative 
    edge applications.
    \item Chapter~\ref{ch:related-work} studies and compares the solutions that have
    been designed in the state of the art of edge storage systems
    to answer both challenges and requirements presented before,
    and discusses the remaining challenges that need to be addressed.
  \end{itemize}
  \item Part~\ref{part:contributions}, in light of what we saw in the existing work, 
  we will here justify some protocol choices used in our approach, 
  which aims to reduce shortcomings observed in some existing systems.
  And present the design, implementation and the programming model of our
  main contribution, \system{}, a database and middleware designed to address
  requirements of group collaboration.
  \item Part~\ref{part:evaluation} provides an experimental evaluation 
  demonstrating the benefits of our approach, compared to other solutions 
  from the state of the art.
  \item the last part~\ref{part:conclusion} we summarize our contribution, and 
  present our vision for the future requirements towards more reliable,
  scalable and secure collaborative edge storage systems.
\end{itemize}
